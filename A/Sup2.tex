% \section*{Tariffs and Quotas}
% 
% \begin{itemize}
% 	\item The equivalence of tariffs and quotas:
% 	\item Exception:
% 	\begin{itemize}
% 		\item \emphc{Uncertainty:} Random shocks shift demand and supply. Import quantity changes with tariffs but not with quota - domestic price more volatile.
% 		\item \emphc{Rent Seeking:} \emph{Quota-rents} imply tariffs preferred over quotas. 
% 	\end{itemize}
% \end{itemize}
% 
% \newpage\newpage

\begin{enumerate}
	\renewcommand{\labelenumi}{Q\,\arabic{enumi}.}
	\renewcommand{\labelenumii}{(\alph{enumii})}
	\item 
	\begin{enumerate}
		\item \emphc{Choosing between tariffs and quotas.} 
		\begin{itemize}
			\item The \emph{equivalence} of tariffs and quotas. 
			\item Tariffs leads to government revenue and quotas lead to producer rent but equivalent.
			\item Tariffs preferred due to the following reasons
			\begin{itemize}
				\item Quota rents can be taken away by foreign licensed exporters if voluntary export restrictions (VER) applied
				\item Rent seeking by domestic producers
				\item Monopoly - with tariffs, the monopolist becomes a price taker, with quota the monopolist uses her market power to obtain rents. Consumer surplus higher with tariffs. 
			\end{itemize}
		\end{itemize} 
		\item \emphc{Imperfectly competitive markets - the logic of export subsidy. }
		\begin{itemize}
			\item Perfect competition - export subsidy unambiguously inefficient (never improve social welfare)
			\begin{itemize}
				\item Export subsidy (shift down the domestic supply curve) and decrease the world price by less than the export subsidy
				\item Exporting country suffers from 1) production distortion 2) consumption distortion and 3) a terms of trade loss (since it is an exporting country)
				\item it does raise the domestic producer surplus by raising the domestic price - interest groups
			\end{itemize}
			\item \textbf{Cournot} -export subsidy can lead to welfare gain - scope for improvement
			\begin{itemize}
				\item GM versus Toyota playing Cournot.
				\item GM subsidy means that GM increases it output (like a stackelberg leader) as its reaction function shift out 
				\item [+] Toyota's \emph{rents shift} to GM (welfare gain for the US)
				\item [$-$] \emph{Terms of trade deteriorate} for US (welfare loss for the US)
				\item For US to benefit, GM profits needs to be taxed and redistributed or at least cover the cost of subsidy.
			\end{itemize}
			\item Bertrand - export subsidy creates not welfare gain - no scope of improvement
			\begin{itemize}
				\item With Bertrand, GM and Toyota are producing at $P=MC$ and not making any profit
				\item Export subsidy can shift profits - GM will take over the whole Japanese market but \emph{benefit to the firm} equal \emph{cost of export subsidy}. No net benefit to the US.
			\end{itemize}
		\end{itemize}
	\end{enumerate}
	\item 
	\begin{enumerate}
		\item \emphc{Tariffs protection for domestic industry} 
		\begin{itemize}
			\item Choosing the tariffs is from purely domestic considerations and it is never efficient for a small economy to choose a tariffs. 
			\begin{align*}
				\tau = \frac{1}{\varepsilon}
			\end{align*}
			This is because, for a small economy $\varepsilon$, the elasticity of the supply tens to infinity and the optimal tariff $\tau$ is the inverse, so it should be zero. 
			\item This is because there would be no terms of trade benefits and there would be production and consumption distortions.
			\item This is static argument.
			\item Dynamic argument - infant industry protection. The supply curve shifts out over time as the country moves from being an importer to an exporter. (This does not work well with the perfect competition assumption as it would need increasing returns to scale, which leads to imperfect competition.)
		\end{itemize}
		\item \emphc{Restriction export of food:}
		\begin{itemize}
			\item Export tax would increase the global price of food.
			\item Compare the benefit in terms of terms of trade and loss in terms of consumer and producer distortions.
			\item For the poor, much better to redistribute after maximising the national welfare than to reduce the prices - i.e., keep the two objectives separate. 
			\item \emph{Stolper-Samuelson:} with land intensive food production, export restrictions should reduce the real return to land (abundant factor) and increase the real return to labour (scarce factor), which would benefit the poor.
		\end{itemize}
		\item \emphc{Gravity Model:}
		\begin{itemize}
			\item 
			\begin{align*}
				T_{ij} = \frac{A Y_{i}^{a}Y_{j}^{b}}{D_{ij}}
			\end{align*}
			where $A,a,b,c$ constants. $Y_{i}$ and $Y_{j}$ size of the economy, $T_{ij}$ trade between $i$ and $j$ and $D_{ij}$ is the distance between $i$ and $j$.
			\item only important factor in the gravity model is the distance
			\item Conversely, Hecksher-Ohlin model tells us that differences in endowments and technology drive trade (more micro founded)
			\item H-O predicts 
			\begin{itemize}
				\item factor price equalisation
				\item redistributional impact of trade - the owners of factors that are abundant gain and owners of factors that are scarce lose
			\end{itemize}
		\end{itemize}
	\end{enumerate}
\end{enumerate}