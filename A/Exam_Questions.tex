\section*{2010}

\begin{enumerate}
\item [7.] 
\begin{enumerate}
	\item [(a)] \emphc{If asked to choose between a quota and its equivalent tariff, most trade economists strongly argue in favour of one over the other. Which one do they prefer, and why?}
	\begin{itemize}
		\item Prefer tariffs to quotas. Tariffs yield government revenue, quotas yield rent, unless they are auctioned off (which happens very rarely). Tariffs do better when demand expands. Tariffs `cap' the ability of a domestic monopolist to charge high prices, so domestic welfare is higher with a tariff, when the domestic market has a monopoly seller. \par\aleblue{Sanjay's Lecture 4.5: Quota versus Tariff.}
	\end{itemize}
	\item [(b)] \emphc{In industries characterised by imperfectly competitive markets, firms sometimes argue for government subsidies, on the grounds that these can shift profits from foreign firms to the domestic firm(s). Further, they suggest, these interventions would raise the welfare not just of the benefiting firm, but of the domestic economy as a whole. Evaluate this argument critically. In particular, even if the basic argument might be correct in specific cases, point out the counter-arguments: why are most trade economists mostly critical of an activist strategic trade policy?}
	\begin{itemize}
		\item Evaluate critically:\ very sensitive to assumptions about nature of competition:\ Bertrand or Cournot. ``Rent-shifting'' can happen, under Cournot competition, but not Bertrand, so a subsidy might make sense in that case.

 The main disadvantage is that it can lead to both ``rent seeking'' and beggar-thy-neighbor policies, which can increase one country's welfare at the other country's expense. Such policies can lead to a trade war in which every country is worse off, even though one country could become better off in the absence of retaliation. This is the danger in enacting strategic trade policy: it often provokes retaliation, which in the long run, can make everyone worse off.
	 \par\aleblue{Sanjay's Lecture 5 and 5.5}
	\end{itemize}

\end{enumerate}

\item [8.]

\begin{enumerate}
	\item [(a)] \emphc{Consider a developing country witnessing an increase in its exports of manufactured products that are intensive in unskilled labour, whose employment and real wages have also been rising correspondingly. The exporters' business association argues that the government should reduce its expenditure on education, for fear that otherwise the country will lose its comparative advantage in these unskilled-labor-intensive exports. Do you agree or disagree? Why?}
	\begin{itemize}
		\item Disagree. The payment of higher wages depends mostly on workers' attainment of higher productivity. (Chapter 3 in the textbook contains a discussion of the differences in productivity and the corresponding wages across countries). Competition in the unskilled-labour-intensive products market naturally constrains wage growth of unskilled labour. If there is an association between education and skills, then the increase in education and productivity will allow the export sector to compete with other low(er) wage countries even as the wages for these workers rise domestically.
	\end{itemize}
	\item [(b)] \emph{[The question on the exam is basically screwed up. Use this one instead].} \emphc{The government of the small country of Velocipedia has identified two-wheeled vehicles as an infant industry worthy of protection, and is contemplating a 10\% tariff on imports of all two-wheeled vehicles. There are two kinds of two-wheeled vehicles: bicycles and scooters, each of which is produced by assembling component kits that are currently freely traded at world prices. Bicycle kits cost \$150, and the world price for bicycles is \$200. Scooter kits cost \$300, and the world price of scooters is \$500. On which of these two vehicles does the 10\% tariff confer greater protection? Remark briefly on the implications for industrial promotion policies.}
	\begin{itemize}
		\item 
		The comparison of the respective Effective Rates of Protection is straightforward.

		$ERP=(V_{d}-V_{w})/V_{w}$

		where $V_{d}$ = Value added at domestic prices

		and $V_{w}$ = Value added at world prices

		For scooters, $V_{w}=500-300=200$, and $V_{d}=500(1+.10)-300=550-300=250$

		Hence ERP(scooters) $=(250-200)/200=25\%$

		Similarly, for bicycles, $V_{w}=200-150=50$, and $%
		V_{d}=200(1+.10)-150=220-150=70$

		Hence ERP(bicycles) $=(70-50)/50=40\%$

		Implications: small level of nominal protection (10\%) can confer a much larger degree of effective protection. Students should point out that what is being protected is not the good per se, but rather the activity (assembly in this case), hence the effective protection is greater for lower value-added sectors, (bicycles in this case) - encourages growth in activities that involve `screwdriver technology'.
	\end{itemize}
	\item [(c)] \emphc{One of the commonly used assumptions in deriving the Heckscher-Ohlin model is that tastes are homothetic, or that if the per capita incomes were the same in two countries, the proportions of their expenditures allocated to each product would be the same as it is in the other country. Imagine that this assumption is false, and that in fact, the tastes in each country are strongly biased in favor of the product in which it has a comparative advantage. How would this affect the relationship between relative factor abundance between the two countries, and the nature (factor-intensity) of the product each exports? What if the taste bias favored the imported good?}
	\begin{itemize}
		\item If in fact national tastes were strongly biased in favor of the product in which the country enjoyed a comparative advantage, then we would expect a bias in favor of rejecting the Heckscher-Ohlin Theorem in actual trade data. The engine driving the H-O model is that a country should be expected to have a relatively low cost of producing the good in which it has a comparative advantage. However, the respective demand forces would tend to raise the price of this good, so that the expected pattern would not generally be observed. However, if the tastes were biased in favor of the imported good, then the predictions of the Heckscher-Ohlin Theorem would be expected to be generally observed.
	\end{itemize}
\end{enumerate}
\end{enumerate}


\newpage
\section{2011a}




\begin{enumerate}
	\item [Q6.] For the purposes of this question, suppose that Airbus is totally owned by European citizens and Boeing is totally owned by US citizens. The two firms engage in Bertrand competition, as price-setting duopolists, in world markets. For the purposes of this question you should ignore the market for airplanes within the EU and the US, and focus only on the export market in the rest of the world. The demand functions for their planes in that market are given by
	\begin{align*}
		Q_{A}=200-P_{A}+{\frac12} P_{B} \\ 
		Q_{B}=200+{\frac12}P_{A}-P_{B}
	\end{align*}
	where the subscripts $A$ and $B$ refer to Airbus and Boeing respectively.
	The average and marginal costs of producing each plane equal 64 in each
	country. The EU is contemplating a strategic trade policy.
	
	\begin{enumerate}
		\item [(a)] Suppose the EU offers Airbus a subsidy $S_{A}$ for each plane it exports. (A negative value of $S_{A}$\ can be interpreted as an export tax).
	\begin{enumerate}
		\item [(i)] Write down expressions for the profits of Airbus and Boeing as functions of the prices set by the two firms, and the EU subsidy. 
		\item [(ii)] Calculate the best response functions giving each manufacturer's profit-maximising price as a function of the other's price and the EU subsidy. 
		\item [(iii)] Solve these functions to express the Bertrand-Nash equilibrium prices as functions of the EU subsidy. 
		\item [(iv)] Find the resulting expressions for the quantities and the profits of each firm, also as functions of the EU subsidy.
		\end{enumerate}
	\item [(b)] The EU wants to maximize its welfare or total surplus, which equals Airbus' profit minus the budgetary cost of the subsidy. The US government is not deploying any strategic trade policy; its total surplus is simply Boeing's profit.
	
	\begin{enumerate}
		\item [(i)] What are the EU and US total surpluses in the absence of any policy, that is, when $S_{A}=0$? 
		\item [(ii)] What is the EU's optimal choice of $S_{A}$? What are the resulting EU and US total surpluses and how do they compare to the case when $S_{A}$ is set at 0? Give an economic explanation for your answer.
		\end{enumerate}
		\end{enumerate}
	
	\newpage


	
	Points to note: 
	\begin{itemize}
		\item Krugman article: high fixed cost of entry relative to market size.
		\item Strategic subsidy gets the new entrant the whole market from the incumbent. So, there would be strategic subsidy from both firms (Countries).
		\item In this question, the market structure is as follows.
		\begin{itemize}
			\item It is Bertrand Competition with differentiated products. If the firms could perfectly collude, then $Q=Q_{A}+Q_{B}$ and $P_{A}=P_{B}$. This means $P=232$ and $Q=168$ and the profits is maximised. The other extreme is pure Bertrand Competition where the two products $A$ and $B$ are perfect substitutes. With Bertrand Competition, the firms will drive down the prices to marginal cost $64$ and each  firm will not retains any rents. Bertrand Competition with differentiated products is some where in the middle where the firms make intermediate profits at intermediate price. 
			\item In this case, increasing prices would be useful but the non-cooperative nature of the strategic interactions makes it difficult. The optimal $S_{A}$ turns out to be a export tax where $S_{A}<0$. This is because $S_{A}<0$ drives us both prices.\footnote{It drives $P_{A}$ more than $P_{B}$} As prices get closer to monopoly prices, the size of the pie increases. Given that $A$'s price rises at a greater rate than $B$'s price, most of the increase in pie goes to $B$. But, $A$ gets some, which gives it an incentive to put the export tax.
			\item Further, what you would get is $S_{B}<0$, the export tax that US would like to impose on $B$. Thus, you would get an reactions functions $S_{A}(S_{B})$ and $S_{B}(S_{A})$. The equilibrium export tax would be $S_{A}, S_{B} \in (-15, 0)$.
			\item Think about the market structure in this question. Natural monopoly like Evian. It would be that the market power comes from increasing returns to scale and a small market size. Explore the relationship between increasing returns to scale and market power. 
		\end{itemize}
	\end{itemize}

	\textbf{Answers}
	
	\bigskip

	(a) (i) Profit expressions:

	$\pi _{A}=[P_{A}-(64-S_{A})][(200+\frac{1}{2}P_{B})-P_{A}]$

	$\pi _{B}=[P_{B}-64][(200+\frac{1}{2}P_{A})-P_{B}]$

	\bigskip

	(ii) Using the result given in the statement of the question, $\pi _{A}$ is
	maximized when

	$P_{A}=\frac{(64-S_{A})+(200+\frac{1}{2}P_{B})}{2}=132-\frac{1}{2}S_{A}+%
	\frac{1}{4}P_{B}.$

	Similarly, $\pi _{B}$ is maximized when

	$P_{B}=\frac{64+(200+\frac{1}{2}P_{A})}{2}=132+\frac{1}{4}P_{A}.$

	\bigskip

	(iii) Solving, by substituting for $P_{B}$,

	$P_{A}=132-\frac{1}{2}S_{A}+\frac{1}{4}(132+\frac{1}{4}P_{A})=132+33--\frac{1%
	}{2}S_{A}+\frac{1}{16}P_{A}$

	or

	$\frac{15}{16}P_{A}=165-\frac{1}{2}S_{A}$,

	or

	$P_{A}=\frac{16\ast 165}{15}-\frac{16}{15\ast 2}S_{A}=176-\frac{8}{15}S_{A}$.

	Then

	$P_{B}=132+\frac{1}{4}[176-\frac{8}{15}S_{A}]=132+44-\frac{2}{15}S_{A}=176-%
	\frac{2}{15}S_{A}.$

	\bigskip

	(iv) Substituting for prices in the demand functions,

	$Q_{A}=200-[176-\frac{8}{15}S_{A}]+\frac{1}{2}[176-\frac{2}{15}S_{A}]=112+%
	\frac{7}{15}S_{A}$

	$Q_{B}=200+\frac{1}{2}[176-\frac{8}{15}S_{A}]-[176-\frac{2}{15}S_{A}]=112-%
	\frac{2}{15}S_{A}$

	and

	$\pi _{A}=[176-\frac{8}{15}S_{A}-(64-S_{A})](112+\frac{7}{15}S_{A})=(112+%
	\frac{7}{15}S_{A})^{2}$

	$\pi _{B}=[176-\frac{2}{15}S_{A}-64](112-\frac{2}{15}S_{A})=(112-\frac{2}{15}%
	S_{A})^{2}.$

	\bigskip

	(b) (i) When $S_{A}=0$,

	$P_{A}=P_{B}=176;Q_{A}=Q_{B}=112;\pi _{A}=\pi _{B}=112^{2}=12544$.

	The EU and US surpluses equal the respective firms' profits, so 12544 each.

	\bigskip

	(ii)The EU's optimal choice of  $S_{A}$ is given by:

	Choose $S_{A}$ to max $W_{EU}=\pi _{A}-S_{A}Q_{A}=(112+\frac{7}{15}%
	S_{A})^{2}-S_{A}(112+\frac{7}{15}S_{A})=(112+\frac{7}{15}S_{A}-S_{A})(112+%
	\frac{7}{15}S_{A})=(112-\frac{8}{15}S_{A})(112+\frac{7}{15}S_{A})$ 

	Taking\  \ the f.o.c., we get:

	$\frac{d}{dS_{A}}[(112-\frac{8}{15}S_{A})(112+\frac{7}{15}S_{A})]=\frac{7}{15%
	}(112-\frac{8}{15}S_{A})-\frac{8}{15}(112+\frac{7}{15}S_{A})=0$

	$\Longrightarrow 7(112-\frac{8}{15}S_{A})-8(112+\frac{7}{15}S_{A})=0$

	$\Longrightarrow 7\ast 112-8\ast 112=\frac{56}{15}S_{A}+\frac{56}{15}S_{A}$

	$\Longrightarrow -112=\frac{112}{15}S_{A}$

	$\Longrightarrow S_{A}=-15$

	With $S_{A}=-15$, we have

	$P_{A}=176+8=184;P_{B}=176+2=178;Q_{A}=112-7=105;Q_{B}=112+2=114$,

	and

	$W_{EU}=\pi _{A}-S_{A}Q_{A}=(112+\frac{7}{15}S_{A})^{2}-S_{A}\ast
	105=(112-7)^{2}+15\ast 105=105^{2}+15\ast 105=120\ast 105=12600,$

	$W_{US}=\pi _{B}=(112+\frac{2}{15}15)^{2}=114^{2}=12996.$

	The EU surplus is higher than it would be without the policy, but the US surplus (Boeing's profit) is higher still. The EU's optimal strategic policy is an export tax, which lowers Airbus's sales and increases Boeing's. In a price-setting duopoly, Airbus benefits if Boeing charges a higher price. Boeing's best response function in prices is upward-sloping. Therefore the way for the EU to get Boeing to charge a higher price is to commit Airbus to charging a higher price, which is done by shifting its price best-response function outward by means of the export tax. Such a policy at the optimal level does increase EU's welfare above its no-policy level (after all, 0 is a feasible choice for the policy, so the optimal choice should do at least as well). But it gives Boeing the advantage in price-setting: Boeing's price rises by only 2 when Airbus's rises by 8 (a 4:1 ratio along the best response curve). Therefore Boeing's profits and the US surplus go up by even more. (The answer might also point out that, in reality, in such a situation the governments playing non-cooperatively may each wait for the other to move, or they may cooperate and cartelize the world market).

 By contrast: [Not necessary for this answer] In a quantity-setting (Cournot) duopoly, Airbus benefits if Boeing offers a smaller quantity for sale. The EU's optimal strategic policy would be an export subsidy, which increases Airbus's sales and lowers Boeing's.The EU surplus would be higher, and the US surplus (Boeing's profit) lower, than they would be in the absence of the policy. The EU's strategic subsidy would yield this outcome by shifting out Airbus's best response function and driving Boeing down its best response function.

\emph{“Rent-shifting” can happen, under Cournot competition, but not Bertrand, so a subsidy might make sense in that case.}
	
	
\end{enumerate}

\newpage

\section*{2011b}


\begin{enumerate}
\item [7.] Suppose that the domestic shirt indutry is protected by an ad
valorem import tariff, denoted by $t_{s}$. The sole intermediate input
required in the production of shirts, in addition to labor, is cloth, on
which there is also an ad valorem tariff, denoted by $t_{i}$.
\begin{enumerate}
	\item [(a)] Suppose that\ the tariff rate on imported shirts, $t_{s}$ is greater than the tariff rate\ $t_{i}$\ on the intermedate input, cloth. Compare the `effective rate of protection' (ERP) with the nominal rate of protection on shirts. 
	\item [(b)] Suppose that trousers production requires the same intermediate input as shirts, and the tariff rates that apply to trousers production are the same as that in shirt production. \ Further, suppose that the world price of trousers is the same as that of shirts. However, the value added in trouser production is greater than that\ in shirt production. Compare the ERP on shirt production and trouser production. Which activity benefits more from the given set of tariffs? Discuss the broader implications of trade protection policies on the likely pattern of industrial development in the domestic economy. 
	\item [(c)] One of the characteristics of globalization is the lengthening of the supply chain, as different parts of the production process are carried out in different locations, often in different countries. Is this fragmentation of the production process likely to magnify the effects of the kinds of protection described above, or to diminish it? Discuss.

\end{enumerate}
\end{enumerate}



\textbf{Answers}

(a) It is straightforward to show that, when $t_{s}>t_{i}$ , then ERP $>$ Nominal rate of protection (NRP), which is just 
$t_{s}$.

$ERP=(V_{d}-V_{w})/V_{w}$

where $V_{d}=(p_{d}-c_{d})$ = $(p_{w}(1+t_{s})-c_{w}(1+t_{i}))=$Value added
at domestic prices

and $V_{w}$ = $(p_{w}-c_{w})$ = Value added at world prices

This can be reduced to: ERP = $\frac{p_{w}t_{s}-c_{w}t_{i}}{p_{w}-c_{w}}=%
\frac{(p_{w}-c_{w})t_{s}+c_{w}(t_{s}-t_{i})}{p_{w}-c_{w}}=t_{s}+\frac{%
c_{w}(t_{s}-t_{i})}{p_{w}-c_{w}}>t_{s}>t_{i}$

Implications: a given level of nominal protection ($t_{s}$) can confer a
much larger degree of effective protection. What is being protected is not
the good per se, but rather the activity (conversion of cloth to shirts and
trousers in this case).

(b) Given: $(p_{t}-c_{t})>(p_{s}-c_{s}).$

$ERP_{s}$=$\frac{p_{t}(1+t_{s})-c_{t}(1+t_{i})-(p_{t}-c_{t})}{(p_{t}-c_{t})}=%
\frac{p_{t}t_{s}-c_{t}t_{i}}{(p_{t}-c_{t})}=\frac{%
p_{t}(t_{s}-t_{i})+(p_{t}-c_{t})t_{i}}{(p_{t}-c_{t})}=t_{i}+\frac{%
p_{t}(t_{s}-t_{i})}{(p_{t}-c_{t})}$

Similarly,

$ERP_{s}$=$t_{i}+\frac{p_{s}(t_{s}-t_{i})}{(p_{s}-c_{s})}.$

To compare the two ERPs, we can use the fact that $p_{s}=p_{t}$ to establish
that $ERP_{s}>ERP_{t}.$

Implication: the effective protection is greater for lower value-added
sectors, (shirts in this case). Hence a given pattern of protection can end
up attracting the lowest value-added industries. This encourages growth in
activities that involve `screwdriver technology'.

(c) This magnifies the effect of protection, since each individual piece of
the production process is a small part of the total value added.\ And the
smaller the value added, the greater the effect of protection. Implications
also for competition between (developing) countries to attract FDI by
offering protection.

\bigskip

\begin{enumerate}
\item[8.] 

	\begin{enumerate}
		\item [(a)] \emphc{Consider a small open economy, too small to affect world prices, that produces as well as imports bicycles. Both the world market and the domestic market are perfectly competitive. Suppose that domestic bicycle manufacturers argue that the world price for bicycles is artificially low, due to the subsidies that foreign manufacturers get from their governments. Hence, they suggest, they should be given tariff protection too. If the trade policy maker for this country wants to maximize social welfare in this economy, should he impose a tariff on bicycle imports? Explain.}
		\begin{itemize}
			\item  No. The optimal tariff for a small open economy is zero. Absent any strategic considerations, the fact that foreign producers are subsidised is irrelevant in the determination of the optimal tariff here.
			 \par\aleblue{Sanjay's Lecture 4 and 4.5. Within Lecture 4, pages 33-37 are extremely important - they deal with} ``What is the optimal tarriff in small and large country'' \par 
			\aleblue{McLaren. Chapter 7.}
		\end{itemize}
		\item [(b)] \emphc{Suppose that a general Gravity Model, with appropriate adjustments for distance, borders, etc., can fit the available trade data fairly closely. What, if anything, can we still learn from more complicated models of trade, such as the Heckscher-Ohlin model?}
		\begin{itemize}
			\item While it might do a good job of predicting the volume and direction of trade, the gravity model has some limitations:\par it does not determine which country produces which goods; intra-industry versus inter-industry trade; how does trade affect factor incomes; how does the political process affect the formation of policy; and more generally, does not provide much of a structured way of thinking about the adjustments required to fit the model to the data.
		\end{itemize}
		\item [(c)] \emphc{A rise in world food prices, as seen in recent years, frequently leads to calls for food-exporting developing countries to consider placing restrictions on exports in order to protect the poor in those countries. Suppose you are asked to provide policy advice on the effects of an export tax (or an outright ban) on food exports from such a country, and to suggest possible alternatives. In particular, describe the likely effects on the price of food, the consumption and welfare of the poor, and national welfare more generally, for the country seeking your advice.}
		\begin{itemize}
			\item [(c)] Obviously, export restrictions will reduce the domestic price of food. The consumption of the poor:\ less straightforward, and depends on the effect on the income of the poor, in addition to their expenditures on food. Stolper-Samuelson suggests that with land-intensive food production, export restrictions should reduce the real return to land, and raise that to labor, which should benefit the poor. Even with specific factors, and the ambiguous effect on wage, the importance of food in the budget of the poor should mean that they benefit.

 Overall (unweighted) welfare will decline, because others (e.g., the owners of land) are adversely affected, and their losses will outweigh the gains to the poor. This implies that there exist direct tax-and-transfers that could be used instead, which would have a smaller negative effect on overall welfare. However these may not be politically or administratively feasible.

 The theory of the second best suggests that other alternatives might do better. E.g., a food consumption subsidy, targeted to the poor, without distorting trade, would still allow the country to benefit from the high prices of their food exports. Production incentives would not be distorted. But subsidies are a visible expenditure from the government budget, so may be difficut to finance.
		\end{itemize}
\end{enumerate}
\end{enumerate}


















