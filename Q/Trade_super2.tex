\documentclass[12pt,onesided]{article}

%\documentclass[a4paper,12pt]{amsart}
%Thursday, November 24, 2005 at 08:57
\usepackage{amsmath, amsthm, amssymb, color, graphicx, graphics} \linespread{1.1}

\setlength{ 
\textwidth}{16cm} \setlength{ 
\textheight}{20cm} \setlength{ 
\evensidemargin}{0cm} \setlength{ 
\oddsidemargin}{0cm} \parindent0cm 
\pagenumbering{roman}

%\usepackage[myheadings]{fullpage}
\usepackage{fancyhdr} 
\pagestyle{fancy} 
\renewcommand{\headrulewidth}{0.2pt} 
\renewcommand{\footrulewidth}{0.2pt} \fancyhead[L]{\textbf{Supervision 2}} \fancyhead[R]{\textbf{International Trade \& Trade Policy}} \fancyfoot[L]{Dr.~Kumar Aniket} \fancyfoot[R]{\emph{ Macroeconomics IIB}} \fancyfoot[C]{\thepage}

%\usepackage{times} %Times Roman ps font
\usepackage{palatino}

%\usepackage{newcent}
%\usepackage{bookman}
%\usepackage{avant}
%\usepackage{helvet} %use helvetica for san-serif
%\usepackage{utopia}
%\usepackage{pandora}
%\usepackage{concrete} %Concrete Roman +  Euler Fonts Maths
%\newcommand{\gr}[1]{\displaystyle{\frac{\Delta {#1}}{#1}}}
%bibliography
\usepackage[round]{natbib}

%\bibliographystyle{plainnat}
\bibliographystyle{apalike}

%\bibliographystyle{econometrica}
\begin{document}

\subsubsection*{Questions} 
\begin{enumerate}
	\item 
	
	% 2010 Question 7 
	\begin{enumerate}
		\item If asked to choose between a quota and its equivalent tariff, most trade economists strongly argue in favour of one over the other. Which one do they prefer, and why? 
		\item In industries characterised by imperfectly competitive markets, Firms sometimes argue for government subsidies, on the grounds that these can shift profits from foreign firms to the domestic firm(s). Further, they suggest, these interventions would raise the welfare not just of the benefiting firm, but of the domestic economy as a whole. Evaluate this argument critically. In particular, even if the basic argument might be correct in specific cases, point out the counter-arguments: why are most trade economists critical of an activist trade policy? 
	\end{enumerate}
	\item 
	\begin{enumerate}
		\item 
		
		% 2011 Question 8 
		Consider a small open economy, too small to affect world prices, that produces as well as imports bicycles. Both the world market and the domestic market are perfectly competitive. Suppose that domestic bicycle manufacturers argue that the world price for bicycles is artificially low, due to the subsidies that foreign manufacturers get from their governments. Hence, they suggest, they should be given tariff protection too. If the trade policy maker for this country wants to maximise social welfare in this economy, should he impose a tariff on bicycle imports? Explain. 
		\item A rise in world food prices, as seen in recent years, frequently leads to calls for food-exporting developing countries to consider placing restrictions on exports in order to protect the poor in those countries. Suppose you are asked to provide policy advice on the effects of an export tax (or an outright ban) on food exports from such a country, and to suggest possible alternatives. In particular, describe the likely effects on the price of food, the consumption and welfare of the poor, and national welfare more generally, for the country seeking your advice. 
		\item Suppose that a general Gravity Model, with appropriate adjustments for distance, borders, etc., can fit the available trade data fairly closely. What, if anything, can we still learn from more complicated models of trade, such as the Heckscher-Ohlin model? 
	\end{enumerate}
	% \item For the purposes of this question, suppose that Airbus is totally owned by European citizens and Boeing is totally owned by US citizens. The two firms engage in Bertrand competition, as price-setting duopolists, in world markets. For the purposes of this question you should ignore the market for airplanes within the EU and the US, and focus only on the export market in the rest of the world. The demand functions for their planes in that market are given by 
	% \begin{align*}
	% 	Q_A = 200 - P_A +\frac{1}{2}\cdot P_B\\
	% 	Q_B = 200 + \frac{1}{2} \cdot P_A- P_B 
	% \end{align*}
	% where the subscripts $A$ and $B$ refer to Airbus and Boeing respectively. The average and marginal costs of producing each plane equal 64 in each country. The EU is contemplating a strategic trade policy. 
	% \begin{enumerate}
	% 	\item Suppose the EU offers Airbus a subsidy $S_A$ for each plane it exports. (A negative value of $S_A$ can be interpreted as an export tax). 
	% 	\begin{enumerate}
	% 		\item Write down expressions for the profits of Airbus and Boeing as functions of the prices set by the two firms, and the EU subsidy. 
	% 		\item Calculate the best response functions giving each manufacturer's profit-maximising price as a function of the other's price and the EU subsidy. 
	% 		\item Solve these functions to express the Bertrand-Nash equilibrium prices as functions of the EU subsidy. 
	% 		\item Find the resulting expressions for the quantities and the profits of each firms, also as functions of the EU subsidy. 
	% 	\end{enumerate}
	% 	\item The EU wants to maximise its welfare or total surplus, which equals Airbus's profits minus the budgetary cost of the subsidy. The US government is not deploying any strategic trade policy; its total surplus is simply Boeing's profits.
	% 	\begin{enumerate}
	% 		\item What are the EU and US total surpluses in the absence of any policy, that is, when $S_A = 0$?
	% 		\item What is the EU's optimal choice of $S_A$? What are the resulting EU and US total surpluses and how do they compare to the case when $S_A$ is set at 0? Give an economic explanation for your answer.
	% 	\end{enumerate}
	% \end{enumerate}
\end{enumerate}


\newpage

\subsubsection*{References} \small{ 
\begin{itemize}
	
	\item [] Paul Krugman, Maurice Obstfeld, and Marc Melitz  (2012). \emph{International Economics: Theory and Policy}. 9th edition, Pearson. 
	\item [] John McLaren (forthcoming). \emph{International Economics: Analysis of Globalization and Policy}. Chapter 3,4,7. 
		\item [] Elhanan Helpman and Paul R. Krugman (1989). \emph{Trade policy and market structure}. MIT Press. Chapter 2,3,4,5,6.
	\item [] Krugman, P. R. (1987). ``Is Free Trade Pass\'{e}?'', \emph{Journal of Economic Perspectives}, 1(2), (Autumn), pp. 131–144.
	\item  [] Gene Grossman and Elhanan Helpman (1994). ``Protection for Sale'', \emph{American Economic Review}. 84(4), pp.833-50.
	
	
	
	
	
	% \item [] Abel, Bernanke and McNabb (1998), \emph{Macroeconomics}, European edition, chapter 4.
	% 
	% \item [] Blanchard (2000), \emph{Macroeconomics}, chapter 16.
	% 
	% \item [] Campbell and Mankiw (1989), ``Consumption, Income and Interest Rates: Reinterpreting the Time-Series Evidence,'' \emph{NBER Macroeconomics Annual}, 185ñ216.
	
	% \item [] Deaton (1992), \emph{Understanding Consumption}, Oxford: Clarendon Press.
	% 
	% \item [] Friedman (1957), \emph{A Theory of the Consumption Function}, Princeton University Press.
	% 
	% \item [] Hall (1978), ``Stochastic Implications of the Life Cycle-Permanent Income Hypothesis: Theory and Evidence,'' \emph{Journal of Political Economy}, 86(6):971ñ987 (December).
	% 
	% \item [] Modigliani (1986), ``Life Cycle, Individual Thrift, and the Wealth of Nations,'' \emph{American Economic Review}, 76(3):297ñ313 (June). 
\end{itemize}
}

%\bibliography{ref-super}
\end{document} 
